% ===================================================================
% PREÂMBULO - Configurações do Documento
% ===================================================================
\documentclass[12pt, a4paper, oneside]{abntex2}

% --- Pacotes Essenciais ---
\usepackage[T1]{fontenc}
\usepackage[utf8]{inputenc}
\usepackage[brazil]{babel}
\usepackage[left=3cm, right=2cm, top=3cm, bottom=2cm]{geometry} % Margens ABNT
\usepackage[alf]{abntex2cite} % Citações ABNT

% --- Pacotes Adicionais para Qualidade ---
\usepackage{graphicx}   % Para inserir imagens
\usepackage{xcolor}     % Para usar cores
\usepackage{listings}   % Para formatar blocos de código
\usepackage{hyperref}   % Para links clicáveis no PDF

% --- Configuração do Bloco de Código (Pacote listings) ---
\definecolor{codegray}{rgb}{0.95,0.95,0.95}
\definecolor{commentgreen}{rgb}{0,0.5,0}
\definecolor{stringpurple}{rgb}{0.5,0,0.5}

\lstdefinestyle{pythonstyle}{
    language=Python,
    backgroundcolor=\color{codegray},
    commentstyle=\color{commentgreen},
    keywordstyle=\color{blue},
    numberstyle=\tiny\color{darkgray},
    stringstyle=\color{stringpurple},
    basicstyle=\footnotesize\ttfamily, % Tamanho da fonte do código
    breakatwhitespace=false,
    breaklines=true,
    captionpos=b,
    keepspaces=true,
    numbers=left,
    numbersep=5pt,
    showspaces=false,
    showstringspaces=false,
    showtabs=false,
    tabsize=2
}

% ===================================================================
% INFORMAÇÕES DA CAPA E FOLHA DE ROSTO
% ===================================================================
\titulo{Análise Preditiva de Preços de Smartphones para o Mercado de 2025}
\autor{Leonardo Pereira Gonçalves \and Henrique Murakami Silva \and Nome do Colega 3 \and Nome do Colega 4 \and Nome do Colega 5}
\instituicao{UEMG - Sistemas de informação
    \par
    Mineração de Dados - Rení Aparecido Norberto Pinto
}

\local{Passos - MG}
\data{\today}

% ===================================================================
% INÍCIO DO DOCUMENTO
% ===================================================================
\begin{document}

% --- Elementos Pré-Textuais ---
\imprimirfolhaderosto

\begin{resumo}
    Este trabalho detalha o desenvolvimento de um projeto de mineração de dados focado na previsão de preços de smartphones, com base em um conjunto de dados simulado para o mercado de 2025. O processo abrange desde a estruturação do ambiente de trabalho colaborativo e a carga inicial dos dados, até a análise preliminar de sua consistência. Utilizando a biblioteca Pandas em um ambiente Google Colab, foi realizada a extração e a primeira inspeção dos dados, identificando as tarefas de limpeza necessárias para as etapas subsequentes. O objetivo final é construir e avaliar um modelo de regressão capaz de estimar o preço de um aparelho a partir de suas especificações técnicas.
    \vspace{\onelineskip}
    \noindent
    \textbf{Palavras-chave}: Mineração de Dados. Aprendizado de Máquina. Previsão de Preços. Regressão. Análise de Dados.
\end{esumo}

% --- Sumário Gerado Automaticamente ---
\tableofcontents
\newpage

% ===================================================================
% CORPO DO TRABALHO
% ===================================================================

\section{Introdução}

O mercado de smartphones é segmentado por diversas faixas de preço, e o valor final de um aparelho é influenciado por uma complexa combinação de fatores como memória RAM, qualidade da câmera, capacidade da bateria, entre outros. Para consumidores e analistas de mercado, entender quais características mais contribuem para o custo de um aparelho é um desafio analítico. O problema central que este projeto busca resolver é: \textbf{quais especificações técnicas são os principais fatores que determinam o preço de um smartphone no mercado previsto para 2025?}

Para investigar esta questão, foi utilizado o conjunto de dados \textit{Global Mobile Prices 2025 Extended}, obtido na plataforma Kaggle \cite{shahzadi2024}, que simula as características de 1000 modelos de smartphones.

\subsection{Objetivos}
Com base no problema definido, os seguintes objetivos foram traçados para guiar o projeto.

\subsubsection{Objetivo Principal}
Desenvolver um modelo de aprendizado de máquina capaz de prever o preço (\texttt{price\_usd}) de um smartphone com base em suas especificações técnicas.

\subsubsection{Objetivos Secundários}
\begin{itemize}
    \item Realizar uma análise exploratória para entender a correlação entre as especificações técnicas e o preço.
    \item Identificar o posicionamento de preço das principais marcas.
    \item Treinar e comparar diferentes modelos de regressão para determinar o de melhor performance.
    \item Gerar insights sobre quais características mais agregam valor a um smartphone.
\end{itemize}

\section{Metodologia}
Esta seção descreve as ferramentas, o conjunto de dados e os procedimentos realizados na fase inicial do projeto, que serviram de base para as etapas de análise e modelagem.

\subsection{Ferramentas e Ambiente}
O projeto foi desenvolvido utilizando a linguagem \textbf{Python 3}. O ambiente de desenvolvimento escolhido foi o \textbf{Google Colab}, por sua facilidade de uso e colaboração. A principal biblioteca utilizada nesta fase inicial foi a \textbf{Pandas}, para manipulação e análise de dados. O controle de versão e o trabalho colaborativo foram gerenciados através de um repositório no \textbf{GitHub}.

\subsection{Processo de Extração, Transformação e Carga (ETL)}
Esta etapa foi focada na extração dos dados e em uma análise preliminar para guiar as tarefas de transformação e limpeza.

\subsubsection{Extração e Carga Inicial}
O primeiro passo consistiu em carregar o arquivo \texttt{Global\_Mobile\_Prices\_2025\_Extended.csv} em um DataFrame do Pandas. O código utilizado para a carga e a primeira inspeção é apresentado abaixo.

\begin{lstlisting}[style=pythonstyle, caption={Código para carga e inspeção inicial dos dados.}, label={lst:codigo-inicial}]
# Importando a biblioteca Pandas
import pandas as pd

# Caminho do arquivo no ambiente Colab
caminho = '/content/Global_Mobile_Prices_2025_Extended.csv'

# Carga do CSV para um DataFrame
df = pd.read_csv(caminho)

# Exibindo informacoes gerais (tipos de dados e contagem de nulos)
df.info()

# Verificando a soma de valores nulos por coluna
print(df.isnull().sum())
\end{lstlisting}

\subsubsection{Análise Preliminar e Direcionamento da Limpeza}
A execução do código (Listagem \ref{lst:codigo-inicial}) revelou que o conjunto de dados possui 1000 registros e 15 colunas, sem nenhum valor ausente (nulo), o que simplifica a etapa de tratamento.

Contudo, a inspeção visual dos dados apontou a necessidade das seguintes tarefas de limpeza e transformação, que serão executadas pelo integrante responsável pela etapa de ETL:
\begin{itemize}
    \item \textbf{Limpeza da Coluna `model`:} A coluna que identifica o modelo do aparelho contém ruídos numéricos (ex: ``A98 111''). Estes devem ser removidos para padronizar os nomes.
    \item \textbf{Transformação de Variáveis Categóricas:} A coluna \texttt{5g\_support}, com valores "Yes" e "No", precisará ser convertida para um formato numérico (0 ou 1) para ser utilizada em modelos matemáticos. Outras colunas, como \texttt{processor}, também deverão ser tratadas (ex: via One-Hot Encoding).
\end{itemize}

\subsection{Implementação da Etapa de ETL}

Após a análise preliminar apresentada anteriormente, foi desenvolvido o processo completo de ETL (Extração, Transformação e Carga), responsável por limpar, padronizar e preparar o conjunto de dados para a análise exploratória e posterior modelagem preditiva. O processo foi realizado utilizando Python e a biblioteca Pandas no ambiente Google Colab.
Como explicado por \cite{tecmundo2025}, o processo de ETL envolve etapas de extração, transformação e carga.


\subsubsection{Transformações Aplicadas}

Com base nos problemas identificados, foram realizadas as seguintes operações:

\begin{itemize}
\item \textbf{Limpeza da coluna \texttt{model}:} Remoção de números aleatórios ao final do nome do modelo, garantindo padronização.
\item \textbf{Conversão da variável \texttt{5g\_support}:} Transformação dos valores "Yes" e "No" para valores numéricos binários (1 e 0).
\item \textbf{Codificação de variáveis categóricas (One-Hot Encoding):} Colunas como \texttt{brand}, \texttt{os}, \texttt{processor} e \texttt{release\_month} foram convertidas em variáveis dummy, tornando-as adequadas para algoritmos de machine learning.
\end{itemize}

\subsubsection{Código Utilizado}

\begin{lstlisting}[style=pythonstyle, caption={Código completo de ETL utilizado para limpeza e transformação dos dados.}]

#Importando as bibliotecas essenciais
import pandas as pd

#1. Extrair
caminho_do_arquivo = '/content/Global_Mobile_Prices_2025_Extended.csv'
df = pd.read_csv(caminho_do_arquivo)

#2. Transformar
#Limpeza da coluna model (remover valores irrelevantes)
df['model'] = df['model'].str.replace(r'\s\d+$', '', regex=True)

#Converter a coluna 5g_support para valores numericos
df['5g_support'] = df['5g_support'].apply(lambda x: 1 if x == 'Yes' else 0)

#One-Hot Encoding para variaveis categoricas
colunas_categoricas = ['brand', 'os', 'processor', 'release_month']
df = pd.get_dummies(df, columns=colunas_categoricas, drop_first=True)

#3. Carga: Salvando o dataset processado
df.to_csv('/content/dados_limpos_para_analise.csv', index=False)
\end{lstlisting}

\subsubsection{Resultado Final}

O processo de ETL gerou o arquivo \texttt{dados\_limpos\_para\_analise.csv}, contendo:

\begin{itemize}
\item variáveis categóricas tratadas e codificadas,
\item nomes de modelos corrigidos,
\item ausência de dados nulos ou inconsistências,
\item estrutura final adequada para treinamento de modelos de machine learning.
\end{itemize}

Esse dataset limpo servirá de base para a etapa de análise exploratória, que será realizada pelo próximo integrante do grupo.

\section{Análise Exploratória de Dados (Em Desenvolvimento)}
\textit{[Esta seção será desenvolvida pelo integrante responsável pela análise exploratória. Aqui serão inseridos gráficos e visualizações para investigar a distribuição dos preços, a correlação entre as variáveis e outros insights extraídos dos dados já limpos.]}

\section{Modelagem e Resultados (Em Desenvolvimento)}
\textit{[Esta seção será desenvolvida pelo integrante responsável pela modelagem. Serão detalhados os algoritmos de regressão escolhidos, o processo de treinamento e teste, e a apresentação dos resultados de performance dos modelos, como o Erro Quadrático Médio (RMSE) e o R².]}

\section{Conclusão}
\textit{[Esta seção será preenchida ao final do projeto, resumindo os resultados, discutindo as limitações do modelo e do dataset, e sugerindo possíveis melhorias ou trabalhos futuros.]}

% ===================================================================
% ELEMENTOS PÓS-TEXTUAIS
% ===================================================================
\bibliography{referencias}

\end{document}